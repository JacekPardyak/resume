\documentclass{article}
\usepackage[T2A]{fontenc}

%Hyphenation rules
%--------------------------------------
\usepackage{hyphenat}
\hyphenation{ма-те-ма-ти-ка вос-ста-нав-ли-вать}
%--------------------------------------
\usepackage[english, russian]{babel}
\begin{document}
 
\tableofcontents

\begin{abstract}
  Это вводный абзац в начале документа.
\end{abstract}
 
\section{Предисловие}
 Этот текст будет на русском языке. Это демонстрация того, что символы кириллицы
 в сгенерированном документе (Compile to PDF) отображаются правильно. Для этого Вы должны установить нужный  язык (russian) и необходимую кодировку шрифта (T2A).

\vskip12pt

\textbf{Этот текст будет на русском языке. Это демонстрация того, что символы кириллицы в сгенерированном документе (Compile to PDF) отображаются правильно.}

\vskip12pt

\textit{Этот текст будет на русском языке. Это демонстрация того, что символы кириллицы в сгенерированном документе (Compile to PDF) отображаются правильно.} 

\section{Математические формулы}
Кириллические символы также могут быть использованы в математическом режиме.

\begin{equation}
  S_\textup{ис} = S_{123}
\end{equation}
\end{document}